\documentclass[
bibfile=references
]{hunterthesis}

\DefineAbreviation{VCS}{Very Cool School}

\usepackage{lipsum}  


\begin{document}

\title{Asymptotic Analysis of Underwater Basket Weaving Techniques}
\author{Mohammed Li}
\date{2022-04-01}
\year{2022}
\department{Computer Science}
\advisor{Maria Nguyen}
\reader{James Kumar}

\dedication{To all future students of this wonderful institution}

\acknowledge{Without the great metropolis of New York City, this work would not be possible.}

\abstract{%
Defining a basket as braid with $n$ cords which tiles a torus, we scrutinize the recursive folding pattern across the topological space.
Setting cord density to the vector-space basis permits comparison between the braid periodicity.
This measure quantifies the asymptotic complexity between $n$ and the braid periodic fixed point.
}

\keywords{LaTeX, document class, masters thesis, Hunter College, basket weaving, underwater, complexity theory, braids, recursion, topology}


\ThesisFrontmatter


\chapter{Introduction}

First and foremost, we will remark that Hunter College is a \Abrev{VCS}.
Before the gibberish ramblings, we must recognize the wonderful work of \cite{lipsum2014}, and it's contributions to the production of entropy.
Surely they also attended a \Abrev{VCS}.


\lipsum[2-6]


\chapter{Logging Statement Level Evolution}

\lipsum[7-8]

\begin{figure}[h!]
  \label{Fig:Logo}
  \caption{Hunter College Logo}
  \includegraphics[width=0.9\textwidth]{Hunter-logo}
\end{figure}

\lipsum[9]

\section{More logs}

\lipsum[10-15]

\begin{table}[h!]
	\centering
    \label{Tab:Numbers}
    \caption{Important numbers to observe}
	\begin{tabular}{||c c c c||} 
		\hline
		Col1 & Col2 & Col2 & Col3 \\ [0.5ex] 
		\hline\hline
		1 & 6 & 87837 & 787 \\ 
		2 & 7 & 78 & 5415 \\
		3 & 545 & 778 & 7507 \\
		4 & 545 & 18744 & 7560 \\
		5 & 88 & 788 & 6344 \\ [1ex] 
		\hline
	\end{tabular}
\end{table}

\lipsum[20]


\chapter{Refactoring to Deferred Execution}

\lipsum[21]

\begin{figure}[h!]
	\centering
    \caption{Hunter College Emblem}
    \label{Fig:Emblem}
    \includegraphics[width=0.5\textwidth]{Hunter-emblem}
\end{figure}

\lipsum[22-40]


\chapter{Conclusion}

\lipsum[41-50]


\ThesisBackmatter


\end{document}
